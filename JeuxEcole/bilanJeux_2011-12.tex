% !TEX encoding = IsoLatin

\documentclass[14pt]{extletter}
\usepackage{geometry}                % See geometry.pdf to learn the layout options. There are lots.
\usepackage{extsizes}                
\geometry{a4paper}                   % ... or a4paper or a5paper or ... 
%\geometry{landscape}                % Activate for for rotated page geometry
%\usepackage[parfill]{parskip}    % Activate to begin paragraphs with an empty line rather than an indent
\usepackage[latin1]{inputenc}
\usepackage[cyr]{aeguill}
\usepackage[francais]{babel}
\usepackage{tikz}
\usepackage{textcomp} 
\usepackage{amssymb}
\usepackage{MnSymbol}
%\usepackage[T1]{fontenc}
\usepackage{lmodern}
\usepackage{array}
\usepackage[dash,dot]{dashundergaps}
\usetikzlibrary{decorations.text}
\geometry{top=1cm, bottom=1cm, left=1cm, right=1cm}
\thispagestyle{empty}

\begin{document}
\begin{center}
\begin{Large}PEL jeux de soci�t� 2011/2012 \end{Large}\\ ~ \\
\end{center}
Pendant cette activit�, nous avons propos� aux enfants participant d'�lire le meilleur des jeux auxquels ils ont jou�s. Pour chaque partie de jeu, l'enfant remplit une case du tableau de sa fiche :
\begin{itemize}
\item il �crit le nom du jeu, 
\item il note le jeu en coloriant les �toiles selon son envie d'y rejouer ou pas,
\item il peut �crire ce qu'il a aim� ou d�test� dans le jeu.\\
\end{itemize}
L'�chelle de notation des jeux est la suivante : \\
$\largestar \largestar \largestar \largestar$ :  Je me suis ennuy� avec ce jeu. Je n'y rejouerai jamais. \\
$\filledlargestar \largestar \largestar \largestar$  :  Jeu pas tr�s rigolo. Je pr�f�re jouer � un autre jeu.\\
$\filledlargestar \filledlargestar \largestar \largestar$ : Jeu moyen. Je ne refuserai pas une partie de temps en temps.\\
$\filledlargestar \filledlargestar \filledlargestar \largestar$ : Bon jeu. J'aime y jouer.\\
$\filledlargestar \filledlargestar \filledlargestar \filledlargestar$ : J'ai toujours envie de jouer � ce jeu. \\~\\

Les enfants ont jou� � \textbf{onze jeux}. Ci dessous, le classement final des diff�rents jeux, ainsi que leur note moyenne.

\begin{enumerate}
\item \textbf{dobble} ( 3,63 )
\item \textbf{halli galli} (3,57 )
\item \textbf{saboteur} (3,57 )
\item \textbf{cartagena} ( 3, 11 )
\item \textbf{ambassadeur} ( 3 )
\item \textbf{qwirkle} ( 2,88 )
\item \textbf{le jeux des mains} ( 2,57 )
\item \textbf{la guerre des moutons} ( 2,55 )
\item \textbf{ouga bouga} ( 2,40 )
\item \textbf{rumis} ( 2,1 )
\item \textbf{pingouins} ( 2 )
\end{enumerate}

Ici, vous pourrez d�couvrir la plupart des jeux de l'activit�, demander des compl�ments d'explications sur les jeux et \LARGE{jouer}.

\end{document}  

 
