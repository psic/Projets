% !TEX encoding = IsoLatin

\documentclass[14pt]{extletter}
\usepackage{geometry}                % See geometry.pdf to learn the layout options. There are lots.
\usepackage{extsizes}                
\geometry{a4paper}                   % ... or a4paper or a5paper or ... 
%\geometry{landscape}                % Activate for for rotated page geometry
%\usepackage[parfill]{parskip}    % Activate to begin paragraphs with an empty line rather than an indent
\usepackage[latin1]{inputenc}
\usepackage[cyr]{aeguill}
\usepackage[francais]{babel}
\usepackage{tikz}
\usepackage{textcomp} 
\usepackage{amssymb}
\usepackage{MnSymbol}
%\usepackage[T1]{fontenc}
\usepackage{lmodern}
\usepackage{array}
\usepackage[dash,dot]{dashundergaps}
\usetikzlibrary{decorations.text}
\geometry{top=0.3cm, bottom=0.3cm, left=0.5cm, right=0.5cm}
\thispagestyle{empty}

\begin{document}
\begin{Large}Pr�nom : \hfill{\color{gray} PEL jeux de soci�t� 2011/2012} \end{Large}\\ ~ \\

Pour chaque partie de jeu que je fais, je remplis une case du tableau suivant :
\begin{itemize}
\item j'�cris le nom du jeu, 
\item je note le jeu en coloriant les �toiles selon que j'ai envie d'y rejouer ou pas,
\item je peux �crire ce que j'ai aim� ou d�test� dans le jeu.\\
\end{itemize}
$\largestar \largestar \largestar \largestar$ :  Je me suis ennuy� avec ce jeu. Je n'y rejouerai jamais. \\
$\filledlargestar \largestar \largestar \largestar$  :  Jeu pas tr�s rigolo. Je pr�f�re jouer � un autre jeu.\\
$\filledlargestar \filledlargestar \largestar \largestar$ : Jeu moyen. Je ne refuserai pas une partie de temps en temps.\\
$\filledlargestar \filledlargestar \filledlargestar \largestar$ : Bon jeu. J'aime y jouer.\\
$\filledlargestar \filledlargestar \filledlargestar \filledlargestar$ : J'ai toujours envie de jouer � ce jeu. \\
\begin{center}
\newcolumntype{P}[1]{>{\raggedleft}m{#1}}
\begin{tabular}{|m{1.8cm} p{16cm}|}
\hline
~&~ \tabularnewline 
Jeu : & \dotuline{\hfill} \begin{LARGE}$\largestar \largestar \largestar \largestar$ \end{LARGE}\tabularnewline 
~ & ~ \tabularnewline 
~& \dotuline{\hfill} \tabularnewline 
~&~ \tabularnewline  
Mon avis :  & \dotuline{\hfill} \tabularnewline 
~&~  \tabularnewline 
~& \dotuline{\hfill} \tabularnewline 
~& ~ \tabularnewline         
\hline
~&~ \tabularnewline 
Jeu : & \dotuline{\hfill} \begin{LARGE}$\largestar \largestar \largestar \largestar$ \end{LARGE}\tabularnewline 
~ & ~ \tabularnewline 
~& \dotuline{\hfill} \tabularnewline 
~&~ \tabularnewline  
Mon avis :  & \dotuline{\hfill} \tabularnewline 
~&~  \tabularnewline 
~& \dotuline{\hfill} \tabularnewline 
~& ~ \tabularnewline   
\hline
~&~ \tabularnewline 
Jeu : & \dotuline{\hfill} \begin{LARGE}$\largestar \largestar \largestar \largestar$ \end{LARGE}\tabularnewline 
~ & ~ \tabularnewline 
~& \dotuline{\hfill} \tabularnewline 
~&~ \tabularnewline  
Mon avis :  & \dotuline{\hfill} \tabularnewline 
~&~  \tabularnewline 
~& \dotuline{\hfill} \tabularnewline 
~& ~ \tabularnewline\hline
\end{tabular}
\end{center}

\newpage
~ \\
Pour chaque partie de jeu que je fais, je remplis une case du tableau suivant :
\begin{itemize}
\item j'�cris le nom du jeu, 
\item je note le jeu en coloriant les �toiles selon que j'ai envie d'y rejouer ou pas,
\item je peux �crire ce que j'ai aim� ou d�test� dans le jeu.\\
\end{itemize}
$\largestar \largestar \largestar \largestar$ :  Je me suis ennuy� avec ce jeu. Je n'y rejouerai jamais. \\
$\filledlargestar \largestar \largestar \largestar$  :  Jeu pas tr�s rigolo. Je pr�f�re jouer � un autre jeu.\\
$\filledlargestar \filledlargestar \largestar \largestar$ : Jeu moyen. Je ne refuserai pas une partie de temps en temps.\\
$\filledlargestar \filledlargestar \filledlargestar \largestar$ : Bon jeu. J'aime y jouer.\\
$\filledlargestar \filledlargestar \filledlargestar \filledlargestar$ : J'ai toujours envie de jouer � ce jeu. \\
\begin{center}

\begin{tabular}{|m{1.8cm} p{16cm}|}
\hline
~&~ \tabularnewline 
Jeu : & \dotuline{\hfill} \begin{LARGE}$\largestar \largestar \largestar \largestar$ \end{LARGE}\tabularnewline 
~ & ~ \tabularnewline 
~& \dotuline{\hfill} \tabularnewline 
~&~ \tabularnewline  
Mon avis :  & \dotuline{\hfill} \tabularnewline 
~&~  \tabularnewline 
~& \dotuline{\hfill} \tabularnewline 
~& ~ \tabularnewline         
\hline
~&~ \tabularnewline 
Jeu : & \dotuline{\hfill} \begin{LARGE}$\largestar \largestar \largestar \largestar$ \end{LARGE}\tabularnewline 
~ & ~ \tabularnewline 
~& \dotuline{\hfill} \tabularnewline 
~&~ \tabularnewline  
Mon avis :  & \dotuline{\hfill} \tabularnewline 
~&~  \tabularnewline 
~& \dotuline{\hfill} \tabularnewline 
~& ~ \tabularnewline   
\hline
~&~ \tabularnewline 
Jeu : & \dotuline{\hfill} \begin{LARGE}$\largestar \largestar \largestar \largestar$ \end{LARGE}\tabularnewline 
~ & ~ \tabularnewline 
~& \dotuline{\hfill} \tabularnewline 
~&~ \tabularnewline  
Mon avis :  & \dotuline{\hfill} \tabularnewline 
~&~  \tabularnewline 
~& \dotuline{\hfill} \tabularnewline 
~& ~ \tabularnewline\hline
\end{tabular}
\end{center}


\end{document}  

 
