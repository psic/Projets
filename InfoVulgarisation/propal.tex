\documentclass[a4paper,11pt]{report}
\usepackage[utf8]{inputenc}
\usepackage[T1]{fontenc}
\usepackage[francais]{babel}


% Title Page
\title{Vulgarisation de l'informatique.}
\author{Jérôme Pierson}





% TODO:
% 
% + écrire le projet
% + Matos : 10 ou 11 PC, un vidéo proj, 11 cables réseau 10 à 15m, un ou deux routeur
% + Trouver un linux pour le enfants (mail, Texte, gestionnaire de fichier, programmation simple ...?)
% + Projet @Mairie, @Conseil Général ...
% + Faire relire le projet à Leila et à Martine
% + Projet @Asus, @HP ...
% + Projet @April et GUILD

\begin{document}
\maketitle

\begin{abstract}
\end{abstract}


\part{Vulgarisation de l'informatique}
\chapter{Introduction}
L'informatique un outil de plus en plus important dans nos vies professionnelles, sociales et au jour le jour. Nous voulons apporter des connaissances de base en informatique visant le fonctionnement d'un ordinateur, son architecture, le fonctionnement d'un réseau informatique et les logiciels les plus utilisés : système d'exploitation, traitement texte et mail. La plupart de ces connaissances seront appréhender à travers des activités pédagogiques participatives et sans ordinateurs. 
\part{Détails des cours}
\chapter{Introduction}



10 Séances de 1h pour 10 personnes
\chapter{Les 10 séances}

\section{Ordinateur ?}

- temps d'échange sur les connaissances informatique de base
- Périphériques
--> Activité découpage collage : nécéssite 10 polycopieé représentant un ordinateur en pièce détaché a recomposé

\section{Architecture}

- Description de l'architecture de base : processeur, carte graphique, mémoire, DD, carte réseau, ... ?)
--> Activité pédagogique simulation d'un ordinateur par groupe de 5 (2 groupes donc 2 ordinateurs) :
    + Le prof donne des calculs simple à faire : 3 calculs à la suite
    + Dans chaque groupe 1 élève processeur, 1 élève affichage/Clavier, 3 élve mémoire
    + Un éléve transmet (fonction clavier en écrivant le calcul sur une feuille) le calcul au processeur (un autre éléve) et l'écrit au tableau (fonction d'affichage)
    + Le processeur effectue le calcul et le dit a la mémoire A (un élève) et à l'affichage
    + Deuxième calcul mémoire B
    + Troisième calcul mémoire C
    + Le prof demande l'addition de A+B+C
    + Calcul du resultat et transmission à l'affichage
    + Un élève joue le role de carte réseau et transmet les résultat d'un ordinateur à l'autre

A revoir...

\section{Architecture 2}

- Les octets
--> Activité pédagogique à trouver

\section{Logiciel}
 
(Machines)
- échange sur les logiciels connus
- Système d'exploitation
- Les grandes fonction : système de fichier, réseau, ...
--> Activité pédagogique sur les ordinateurs , découverte libre, dessin

\section{Utilisation du traitement de texte}
 
(Machines)

\section{Internet}

- Principe de base : IP, DNS, divers protocole
--> Simulation d'un réseau (utilisation de fil de laine)
  + IP et DHCP
  + HTTP et mail 

... A affiner ...

\section{Réseau et Internet TP}
 
(Machine)
- En réseau local faire de l'internet
- du mail
- de la messagerie jabber

\section{Logiciel spécifique}
 
(Machines)

\section{Programation}
 
(Machines)

\section{Programation}
 
(Machines)













\end{document}
